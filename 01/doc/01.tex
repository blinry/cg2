\documentclass[a4paper]{scrartcl}
\usepackage[ngerman]{babel}
\usepackage[utf8]{inputenc}
\usepackage[T1]{fontenc}

\usepackage{amsfonts}
\usepackage{amsmath}


\title{Echtzeit-Computergrafik SS 2012}
\subtitle{Lösungen von Aufgabe 1}
\author{\small Marc Kassubeck (4054946), Torsten Thoben (4054959), Sebastian Morr (4109354)}
\date{\today}

\begin{document}

\maketitle

\section*{1.1}

\[
	v_0 \circ v_1 = \Vert a\Vert \cdot \Vert b\Vert \cdot \cos \alpha
	\Rightarrow \cos \alpha = \frac{v_0 \circ v_1}{\Vert a\Vert \cdot \Vert b\Vert}
	\Rightarrow \alpha = \arccos \left(\frac{v_0 \circ v_1}{\Vert a\Vert \cdot \Vert b\Vert}\right)
\]

\[
	\alpha = \arccos \left( \frac{4 + 4 + 1}{9 \cdot 20.25} \right) \arccos \frac{4}{81} \approx 87.1694^\circ
\]

\section*{1.2}


\[
	v_2 = \begin{pmatrix}
	1 \\ 2 \\ -1
	\end{pmatrix}, v_r = \begin{pmatrix}
	0 \\ 0 \\ 1
	\end{pmatrix}, v_3 = v_2 \times v_r = \begin{pmatrix}
	2 \\ -1 \\ 0
	\end{pmatrix},
	v_4 = v_2 \times v_3 = \begin{pmatrix}
	-1 \\ -2 \\ -5
	\end{pmatrix}
\]

Es gibt vier Möglichkeiten, wenn $v_3$ in der $xy$-Ebene liegen soll:\\
\[
	v_3 = \begin{pmatrix} 2 \\ -1 \\ 0 \end{pmatrix}, v_4 = \begin{pmatrix} -1 \\ -2 \\ -5 \end{pmatrix};\
	v_3 = \begin{pmatrix} 2 \\ -1 \\ 0 \end{pmatrix}, v_4 = \begin{pmatrix} 1 \\ 2 \\ 5 \end{pmatrix}\]
\[
	v_3 = \begin{pmatrix} -2 \\ 1 \\ 0 \end{pmatrix}, v_4 = \begin{pmatrix} -1 \\ -2 \\ -5 \end{pmatrix};\
	v_3 = \begin{pmatrix} -2 \\ 1 \\ 0 \end{pmatrix}, v_4 = \begin{pmatrix} 1 \\ 2 \\ 5 \end{pmatrix}
\]

\section*{1.3}
\[
	\begin{pmatrix}
	-1 & 0 & 0\\
	0 & 1 & 0 \\
	0 & 0 & 1
	\end{pmatrix} \cdot
	\begin{pmatrix}
	1 \\ 2 \\ 3
	\end{pmatrix}=
	\begin{pmatrix}
	-1 \\ 2 \\ 3
	\end{pmatrix}
\]
Diese Matrixmultiplikation bewirkt eine Spiegelung an der $yz$-Ebene.

\[
	\begin{pmatrix}
	1 & 0 & 0\\
	0 & 0 & 1 \\
	0 & -1 & 0
	\end{pmatrix}\cdot
	\begin{pmatrix}
	1 \\ 2 \\ 3
	\end{pmatrix}=
	\begin{pmatrix}
	1 \\ 3 \\ -2
	\end{pmatrix}
\]

Dies bewirkt eine Drehung des Vektors um $90^\circ$ um die $x$-Achse (gegen den Uhrzeigersinn).

\[
	\begin{pmatrix}
	2 & 0 & 0\\
	0 & 2 & 0\\
	0 & 0 & 2
	\end{pmatrix}\cdot
	\begin{pmatrix}
	1 \\ 2 \\ 3
	\end{pmatrix}=
	\begin{pmatrix}
	2 \\ 4 \\ 6
	\end{pmatrix}
\]
Diese Matrix verursacht eine Skalierung des Vektors in der Länge mit dem Faktor 2.

\[
	\begin{pmatrix}
	1 & 0 & 0 & 5\\
	0 & 1 & 0 & 0\\
	0 & 0 & 1 & 3\\
	0 & 0 & 0 & 1
	\end{pmatrix}\cdot
	\begin{pmatrix}
	1 \\ 2 \\ 3 \\ 1
	\end{pmatrix}=
	\begin{pmatrix}
	6 \\ 2 \\ 6 \\ 1
	\end{pmatrix}
\]

Die Matrix verschiebt den Vektor um

\[
	\begin{pmatrix}
	5 \\ 0 \\ 3 \\ 0
	\end{pmatrix}
\]

mal der letzten Vektorkomponente.

\end{document}
